\section{密度聚类和加速密度聚类的复杂性分析\label{Appendix:A}}
考虑对$N$个数据点进行聚类, 需要计算任意两点之间的距离, 得到截止局部密度$\rho_i$. 因此需要的运算总数为
\begin{equation}
    N+(N-1)+(N-2)+\cdots+3+2+1=\frac{N(1+N)}{2}
    \label{equ:A-1}
\end{equation}
可以看到, 其复杂度为$O(N^2)$. 然后, 我们需要将一个数据点的局部密度与其他所有点的局部密度进行比较, 得到最小距离$\sigma_i$. 为了加快最小距离的计算, 我们使用了归并排序(\cite{Satish2010})对局部密度的序列进行了排序, 这个操作的复杂度是$O(N\log N)$. 另外, 只有对局部密度高于当前点的数据点才需要进行比较, 因此, 需要的次数与公式~\ref{equ:A-1} 相同, 即为$O(N^2)$. 乘积$\beta$需要$N$次的乘法. 对$\beta$进行归并排序, 选择聚类中心, 复杂度为$O(N\log N)$. 经过$N$次减法, 对排序后的$\beta$进行求梯度运算, 检测其突变. 确定聚类中心后, 进行对数据点进行分配, 复杂度为$O(N)$, 因此, 密度聚类的计算复杂度近似为$O(N^2)$. 

加速密度聚类算法从复杂度为$O(N)$的对直方图进行处理开始, 将直方图的格点作为一个新的数据集进行聚类. 对于二维的情况, 周长数为$M1,M2$, $M1,M2$分别为时间和偏移维度, $M1, M2$远小于$N$. 然后对新数据集进行密度聚类, 得到聚类中心, 这一步骤的复杂度为$O((M1M2)^2)$, 每个聚类中心位置更新是通过重新计算组距内密度最高的点来完成的. 粗略估计每个聚类中心的组距中的数据点数量为$1/K(r_M/r_{clu})^2N$, 其中$K$为聚类中心的个数, $r_M$和$r_{clu}$分别为直方图中组距的平均半径和簇的平均半径, 更新过程中需要对每个聚类中心进行距离计算. K个聚类中心的距离计算总数为
\begin{equation}
    K \sum_{i=1}^{\frac{1}{K}(\frac{r_M}{r_{clu}})^2N} i=\frac{1}{2}\left[\frac{1}{K}\left(\frac{r_{M}}{r_{\mathrm{clu}}}\right)^{2} N+1\right]\left(\frac{r_{M}}{r_{\mathrm{clu}}}\right)^{2} N
\end{equation}

这个更新聚类中心过程的复杂度为$O(1/K$ $(r_M/r_{clu})^4N^2)$. 而最后一步是用最近邻原则将每个原始数据点分配到更新后的聚类中心, 其复杂度为$O(N)$. 通过省略低阶项, 得到加速密度聚类的近似复杂度$O(1/K(r_M/r_{clu})^4N^2)$. 

加速的密度聚类算法可以比原算法快$K(r_{clu}/r_M)^4$倍. 在``S"型数据集的例子中, 有 $5000$ 个数据点, $15$ 个聚类中心. 速率$r_{clu}^2/r_M^2$约为$2$, 因此, 大约比原算法提速了$60$倍. 
