\section{讨论}
为了从局部同相轴斜率中找到速度的极大似然解, 计算出的局部属性被用混合分布模型表示. 对于地震反射事件, 计算得到的速度不确定性直接影响了聚类结果高斯分布的标准差, 其均值代表高斯分布的均值. 高斯分布主要用于理论分析, 用密度聚类得到聚类中心, 不需要明确计算高斯分布的参数. 加速密度聚类算法可以处理不是特别复杂的非球形和峰值重叠严重的数据, 因此, 混合分布模型并不需要严格的高斯分布. 另外, 直方图函数的组距是一个较为稳健参数, 对组距的设置典型值是沿时间轴设置为一个小波长度, 沿偏移轴设置为50到200米. 在加速密度聚类算法中, 不需要聚类中心数量的先验信息. 人们可以将加速密度聚类应用于地震属性分析, 通过设置结点速度的不确定度界限, 可以抑制噪声和多重事件的影响. 该方法的另一个潜在好处是在斜率估计之前就消除了地震道集上的噪声和多重事件. 之后, 还可以将该方法推广到各向异性介质(\cite{Alkhalifah2000,Alkhalifah2000a,Casasanta2011})、深域速度估计和三维情况的应用. 

斜率估计对所提出的方法的性能起决定型作用, 但在多重到达、弱事件和低信噪比的情况下, 斜率的估计仍比较困难, 这也解释了算法的结果在真实数据中为何不是那么完美. 当对速度进行插值选取时, 通过加入对角权重矩阵, 能更好地使用数据的幅度和结点速度处的局部密度(\cite{Fomel2003}), 这种方法所需要的是在目标函数中加入一个对角权重矩阵$W$, 即将目标函数变成
\begin{equation}
    \left\|\mathbf{W}\left(\mathbf{G} \mathbf{v}_{\mathbf{g r i d}}-\mathbf{v}_{\mathbf{k n o t}}\right)\right\|_{2}^{2}+\varepsilon^{2}\left\|\mathbf{L} \mathbf{v}_{\mathbf{g r i d}}\right\|_{2}^{2}
\end{equation}

将加权后的目标函数最小化的解为
\begin{equation}
    \mathbf{v}_{\mathbf{g r i d}}=\left(\mathbf{G}^{T} \mathbf{W}^{2} \mathbf{G}+\varepsilon^{2} \mathbf{L}^{T} \mathbf{L}\right)^{-1} \mathbf{G}^{T} \mathbf{W}^{2} \mathbf{v}_{\mathbf{k n o t}}
\end{equation}

不过, 这种方法仍是一种时域速度估计方法. 对于横向速度变化较大的情况, 可能不太适用. 

