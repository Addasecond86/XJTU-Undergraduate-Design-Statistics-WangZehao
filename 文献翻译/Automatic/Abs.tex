\begin{abstract}
    时域速度和\rr{动校正参数(moveout parameters)}可以直接从\rr{同相轴局部斜率(local event slopes)}中获得, 而\rr{同相轴局部斜率}是基于叠前地震道集估计出来的. 在实际应用中, 估计出的局部斜率总是有一定的误差, 特别是在信噪比较低的情况下. 因此, 地下速度信息可能隐藏在速度和其他\rr{动校正参数}所决定的图像中. 我们开发了一种加速聚类算法, 可以在事先不知道聚类中心数量的情况下寻找聚类中心. 首先, 实施\rr{平面波解构(plane-wave destruction)}来估计\rr{同相轴局部斜率}. 对于地震道集中的每一个样本, 我们根据\rr{同相轴局部斜率}获得速度的估计和它在图像域中的位置. 这些被映射到新的空间中的点展现出了不同类的结构. 我们用混合分布模型来对这些点进行建模. 然后, 对混合模型的聚类中心进行确定, 这些聚类中心就对应着主要地下构造的极大似然速度. 利用估算的速度的不确定约束来选择对应的反射中心. 最后, 对聚类后的采样不均匀的结速度进行插值, 在规则格网上建立高效的速度模型. 在人工合成数据和实际数据上实施该算法, 我们确定了所提出它可以给出精度相对较高的叠加速度模型和时偏移速度模型. 
\end{abstract}