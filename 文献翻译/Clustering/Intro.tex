\section{简介}
速度估计是地震反问题中的一项重要任务, 它通常是从用不同的时差速度对CMP道集进行扫描开始的. 但手动选取速度谱上的能量峰值点是一个十分费力的任务, 只有有经验的专业数据处理员才能完成. 在过去的几十年里, 人们采取了各种方法来完成自动的速度分析. \citeauthor{Toldi1989}(\citeyear{Toldi1989}) 就提出了一种最早的自动速度分析算法, 它将速度表示为可能的速度区间. 通过沿着时差曲线将堆积功率最大化, 从而获得最佳的速度. 相似谱上的自动速度选取问题也可以看作是一个射线跟踪问题. 将相似谱上的能量(\cite{Fomel2009})作为慢度, 则相似谱顶部到底部具有最小移动时间的射线路径就对应一个最佳的选取速度. 另外, 为了克服在CMP道集上的速度分析中的反射分散问题, 可以用包括各向异性在内的速度延续来建立时偏移速度模型(\cite{Adler2002,Fomel2003,Alkhalifah2011,Burnett2011}). 还有一种相似的方法, 叫作图像波传播(\cite{Schleicher2008}). 用不同速度的来重复时偏移(\cite{Yilmaz2001a})也是一种可选方法. 

另一种方法是使用局部同相轴斜率, 在地震道集上接收到的的局部同相轴斜率包含了珍贵的地下信息. 动校正(NMO)速度和时偏移速度可以直接从局部同相轴斜率中得出. \citeauthor{Ottolini1983}(\citeyear{Ottolini1983})提出了与速度无关的局部同相轴斜率成像方法. \citeauthor{Fomel2007}(\citeyear{Fomel2007})和\citeauthor{Cooke2009}(\citeyear{Cooke2009})发现了利用迭前反射数据估计出的局部同相轴斜率可以完成几乎所有常见的时域成像和速度估计任务. 另外, 地震数据或地震图像中包含的局部同相轴斜率和其他局部时差属性也可用于深度域上的速度模型反演, 如立体成像(\cite{Billette1998,Lambare2008})等. 

局部同相轴斜率估计的好坏决定了基于此斜率成像和反演的结果的质量. 局部倾斜叠加(\cite{Ottolini1983})是提取局部同相轴斜率的常用方法之一. 同时, 希尔伯特变换(\cite{Barnes1996,Cooke2009,Zhang2013,Wang2015})也可以用来得到等价于局部同相轴斜率的相位移. 另一种具有稳定性的斜率估计算法是平面波解构法(PWD)(\cite{Fomel2002, Schleicher2009}), PWD通过平面波来逼近局部波场. Chen等人(\cite{Chen2013,Chen2013a})通过最大限度地扁平化分数延迟滤波器, 进一步加速了PWD, 并使其适用于陡峭层. 但由于估计出的局部同相轴斜率易受到噪声和干扰事件的影响而退化, 所以直接用估计出的斜率进行下一步工作可能会导致成像部分受到高频振荡的污染, 从而使得从局部同相轴斜率得出的速度对噪声很敏感. 虽然用图像扭曲的速度图可以得到与局部同相轴斜率相关的时域速度(\cite{Fomel2007}), 但当地震道集中包含多个散射或多路径事件时, 速度图就会变得相当复杂, 从而对速度的求解造成困难. 另外, 速度图上的能量可以作为慢度, 从而通过求解初至射线追踪问题来进行自动速度选取. 除此之外, \citeauthor{Cooke2009}(\citeyear{Cooke2009})还在速度图上使用静音和平均滤波器来抑制多路径事件. 

本文提出了一种基于在局部同相轴斜率计算而来的局部属性上进行聚类的时域速度自动估计方法, 在这其中, 局部属性由一个混合分布模型来表示, 混合分布模型的聚类中心对应于主要地下构造的极大似然速度. 同时, 这些聚类中心对由噪声、干扰事件和多路径事件引起的同相轴局部斜率退化具有鲁棒性. 另外, 我们开发了一种加速聚类算法, 可以高效率地寻找聚类中心, 它通过在聚类后的不均匀采样中心上进行最小二乘插值来建立高效的规则格点上的速度模型. 最早, 通过在人工合成和实际数据上进行验证, 得出了这个算法的性能符合我们理论预期的结论. 
