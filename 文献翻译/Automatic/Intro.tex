\section{简介}
速度估计是地震反问题中的一项重要任务. 宏模型的建立通常是从用不同的动校正速度对CMP道集进行扫描开始的. 手动选取速度谱上的能量峰值点是一个十分费力的过程, 需要有经验的专业的数据处理员才能完成这个任务. 在过去的几十年里, 人们采取了各种方法来完成自动的速度分析. (\cite{Toldi1989})描述了一种最早的自动速度分析算法, 其中速度模型是由可能的区间速度表示的. 沿着动校正曲线将\rr{堆积功率(stacking power)}最大化, 从而获得最佳的速度模型. \rr{相似谱(semblance spectra)}上的自动速度选取问题也可以看作是一个射线跟踪问题. 将相似谱上的能量(\cite{Fomel2009})作为慢度, 则相似谱顶部到底部具有最小移动时间的射线路径就对应一个最佳的选取速度. 为了克服在CMP道集上的速度分析中的反射点分散问题, 可以通过包括各向异性在内的速度延续来建立时偏移速度模型(\cite{Adler2002,Fomel2003,Alkhalifah2011,Burnett2011}). 还有一种叫作\rr{图像波传播(image wave propagation)}的类似的方法(\cite{Schleicher2008}). 不同速度的重复时偏移(\cite{Yilmaz2001a})也是一种替代方法. 

另一种方法是使用\rr{同相轴局部斜率}. 在地震道集上接收到的的\rr{同相轴局部斜率}包含了珍贵的地下信息. 动校正速度(NMO)和时偏移速度可以直接从\rr{同相轴局部斜率}中得出. (\cite{Ottolini1983})提出了与速度无关的\rr{同相轴局部斜率}成像. (\cite{Fomel2007})和(\cite{Cooke2009})表明, 利用叠前反射数据估计的\rr{同相轴局部斜率}可以完成几乎所有常见的时域成像任务和速度估计. 地震数据或地震图像中包含的\rr{同相轴局部斜率}和其他局部时差属性也可用于深度域上的速度模型反演, 如立体成像(\cite{Billette1998,Lambare2008})等. 

\rr{同相轴局部斜率}估计的好坏决定了基于\rr{同相轴局部斜率}的成像和反演结果. \rr{局部倾斜叠加}(\cite{Ottolini1983})是提取\rr{同相轴局部斜率}的标准工具之一. 希尔伯特变换(\cite{Barnes1996,Cooke2009,Zhang2013,Wang2015})也可以用来得到等于\rr{同相轴局部斜率}的相位移. 另一种具有鲁棒性的斜率估计算法是\rr{平面波解构(plane-wave destruction)}(PWD)(\cite{Fomel2002, Schleicher2009}). PWD通过平面波来逼近局部波场. 通过最大限度上扁平化分数延迟滤波器, Chen等人(\cite{Chen2013,Chen2013a})进一步加速了PWD方法, 并使其适用于陡峭层. 由于估计的\rr{同相轴局部斜率}总是受到噪声和干扰事件的影响而退化, 所以直接映射算子可能会导致成像部分受到高频振荡的污染, 从\rr{同相轴局部斜率}得到的速度对噪声很敏感. 虽然图像扭曲的速度图可以得到与\rr{同相轴局部斜率}相关的时域速度, (\cite{Fomel2007})但当地震道集中包含多个散射或多路径事件时, 速度图就会变得相当复杂. 速度图上的能量可以作为慢度, 从而通过求解初至射线追踪问题来进行自动速度选取. (\cite{Cooke2009})在速度图上使用静音和平均滤波器来抑制多路径事件. 

本文提出了一种基于从\rr{同相轴局部斜率}映射来的局部属性聚类的时域速度自动估计方法. 局部属性由一个混合分布模型来表示. 混合分布模型的聚类中心对应于主要地下构造的极大似然速度. 这些聚类中心对由噪声、干扰事件和多路径事件引起的\rr{同相轴局部斜率}退化具有鲁棒性. 我们开发了一种加速聚类算法, 从而高效率地找到聚类中心. 在聚类后的不均匀采样中心上进行最小二乘插值, 建立规则格网上的高效速度模型. 通过在人工合成和实际数据上实施, 我们评估了所提出的算法的性能, 并证实了它符合我们理论上的预期. 
