\begin{abstract}
    地震反演问题中的时域速度和时差参数可以直接从局部同相轴斜率中获得, 而局部同相轴斜率可以从迭前地震道集中估计出来. 但在实际应用中, 估计出来的局部斜率总是有一定的误差, 特别是在信噪比较低的情况下. 因此, 地下速度信息可能隐藏在由速度和时差参数所确定的图像之中. 我们开发了一种加速聚类算法, 可以在事先不知道聚类中心数量的情况下来确定聚类中心. 这个算法分为以下几步, 首先, 用平面波解构算法来估计局部同相轴斜率. 对于地震数据中的所有单个道集, 我们用上一步得到的斜率估计出时域速度它在图中的位置. 而这两个新的数据在分布上则表现出成不同团的特点, 于是, 我们进一步用混合分布模型来对其进行建模. 再进一步, 用聚类算法确定混合模型的各个聚类中心, 而这些聚类中心就对应着主要地下构造的极大似然速度. 再人为设定一个速度不确定性的限度, 根据这个限度来选择对应于某次反射的那个中心. 最后, 对聚类的采样不均匀的结点速度进行插值, 从而建立了性能良好的在规则格点上的速度模型. 我们在人工合成和实际数据上对算法进行了实验, 其结果表明了它可以建立出精度相对较高的叠加速度模型和时偏移速度模型. 
\end{abstract}